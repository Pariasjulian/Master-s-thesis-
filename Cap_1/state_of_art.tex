\section{State of art}\label{sec:stateofart}


In recent years, numerous wireless systems for EEG data acquisition have been developed, with two main approaches standing out: conventional remote monitoring systems and portable smart systems. The former simply digitize the EEG signals and transmit them to a remote unit for processing, usually in a deferred manner \cite{arpaia2020wearable}. On the other hand, portable systems preprocess the signals on a local device, such as a microcontroller (MCU), and wirelessly transmit the data using low-power consumption protocols. \ref{table:bci_hardware} This latter approach is crucial for real-time applications, where low latency is essential.

Marker synchronization in portable EEG acquisition systems \cite{razavi2022opensync}, particularly in applications combined with serious games \cite{damavsevivcius2023serious}, faces several technical and operational challenges. One of the main issues lies in latency in data transmission protocols. In portable EEG systems, precise synchronization between brain events and interactions in the game is crucial \cite{gomezromero2024implications}, but inherent limitations of portable acquisition systems, such as latencies in data transfer protocols, can cause temporal mismatches. These latencies primarily stem from bandwidth constraints in wireless transmission and the need to process large volumes of data in real-time \cite{he2023diversity}.

The type of electrode \cite{liu2023feature} and the number of channels \cite{abdullah2022eeg}  are determining factors in the quality of the data acquired in portable systems. Although dry electrodes offer greater portability, they tend to generate lower-quality signals due to reduced conductivity, which can complicate precise synchronization with other devices, such as serious games. On the other hand, the use of systems with \textbf{low channel density (e.g., 8-16 channels)} \cite{allouch2023effect} is a common strategy in these portable systems to minimize size and improve portability. However, low channel density can affect the spatial resolution of EEG data, limiting the ability to perform accurate analysis of brain patterns. This challenge is reflected in the need to optimize sampling \cite{zheng2023effects}  and data transfer protocols \cite{bayilmics2022survey}  to ensure that captured signals are transmitted efficiently without significant information loss

The sampling rate is another critical factor, as it directly affects the temporal resolution of EEG signals. The combination of low channel density and insufficient sampling rate can make it difficult to capture fast brain events, such as attention shifts, which are essential in applications like serious games. Furthermore, Signal Front-End Amplifiers (AFE) \cite{devi2022survey} play a key role in signal quality. While low-cost AFEs may be suitable for portable systems, they tend to have limitations in processing capacity, which impacts data synchronization by generating noise and distortions in the EEG signals, especially when connected to mobile devices with lower processing power.

Battery life \cite{niso2023wireless} is a significant constraint for portable systems that require long monitoring sessions. EEG systems that operate for several hours often need to optimize their energy consumption, which may involve reducing the sampling rate or channel density, once again impacting data quality and real-time synchronization.


\begin{table}[H]
    \caption{Acquisition devices used for BCI. The table provides an overview of the different hardware devices, their specifications, and communication protocols.}
    \label{table:bci_hardware}
    \centering
    \footnotesize
    \begin{tabularx}{\textwidth}{|X|p{1.8cm}|p{1.8cm}|c|>{\centering\arraybackslash}X|c|X|c|}
        \hline
        \textbf{Hardware BCI} & \textbf{Empresa} & \textbf{Tipo de Electrodo} & \textbf{Canales} & \textbf{Frecuencia de Muestreo} & \textbf{AFE} & \textbf{Protocolo y Transferencia} & \textbf{Batería} \\ \hline
        Cyton + Daisy \cite{OpenBCI_CytonDaisy} & OpenBCI & Flexible / Húmedo / Seco & 16 & 250 Hz - 16 kHz & ADS1299 & RF / BLE / Wi-Fi & 8 h \\ \hline
        actiCAP \cite{BrainProducts_ActiCap} & Brain Products GmbH & Flexible / Húmedo / Seco & 16 & 256 Hz - 16 kHz & - & USB & 16 h \\ \hline
        EPOC X \cite{Emotiv_EPOCX} & Emotiv & Rígido / Húmedo & 14 & 128 Hz & - & BLE / Bluetooth & 6--12 h \\ \hline
        Diadem \cite{Bitbrain_Diadem} & Bitbrain & Rígido / Seco & 12 & 256 Hz & - & Bluetooth & 8 h \\ \hline
        g.Nautilus \cite{Gtec_GNautilusProFlexible} & g.tec & Flexible & 8 / 16 / 32 & 250 Hz & ADS1299 & Propietario & 10 h \\ \hline
        Plataforma para EEG ambulatorio \cite{pinho2014wireless} & - & Activo / Seco & 32 & 250 Hz- -1 kHz & ADS1299 & Wi-Fi 802.11 b/g/n & 26 h \\ \hline
        Sistema para neurofeedback \cite{Totev2023} & - & Pasivo / Seco & 40 & 250 Hz & ADS1298 & RF & - \\ \hline
        BEATS \cite{Beats} & - & Flexible / Húmedo & 32 & 4 kHz & ADS1299 & Wi-Fi & 24 h (cableado) \\ \hline
    \end{tabularx}
\end{table}




In the field of brain-computer interfaces (BCIs), several devices have been developed, each with unique features tailored to specific use cases such as clinical research, neurofeedback, or consumer applications. The Cyton + Daisy system by OpenBCI \cite{OpenBCI_CytonDaisy} supports up to 16 channels and offers a wide sampling rate range of 250 Hz to 16 kHz, making it suitable for high-resolution EEG acquisition. The device uses flexible, wet, or dry electrodes and incorporates the ADS1299 AFE for high-quality signal conversion. It supports data transfer via RF, Bluetooth Low Energy (BLE), and Wi-Fi, allowing for versatile connectivity. With a battery life of 8 hours, this system is highly adaptable, suitable for both research and practical applications in various environments. Another system, actiCAP \cite{BrainProducts_ActiCap} by Brain Products GmbH, features flexible, wet, or dry electrodes and is capable of recording up to 16 channels with a sampling rate range from 256 Hz to 16 kHz. The actiCAP does not use a dedicated AFE and instead relies on a USB protocol for data transfer. The device provides a robust 16-hour battery life, making it an ideal choice for long-duration experiments and clinical settings that require stable signal acquisition over extended periods. The EPOC X \cite{Emotiv_EPOCX} by Emotiv is a more compact and consumer-oriented BCI device that uses rigid, wet electrodes and supports 14 channels with a sampling rate of 128 Hz. This device employs Bluetooth Low Energy (BLE) for wireless data transfer, and its battery life ranges from 6 to 12 hours, depending on usage. While its lower sampling rate may limit its use for high-resolution research, the EPOC X remains a popular choice for applications in neurofeedback, cognitive training, and general user interaction. The Diadem \cite{Bitbrain_Diadem} system by Bitbrain uses rigid, dry electrodes and supports 12 channels with a sampling rate of 256 Hz. It operates via Bluetooth for data transmission and has a battery life of 8 hours, providing a balance between portability and signal quality. The g.Nautilus \cite{Gtec_GNautilusProFlexible} system by g.tec offers great flexibility, supporting configurations with 8, 16, or 32 channels. It operates at a sampling rate of 250 Hz and uses the ADS1299 AFE for high-performance signal acquisition. The system is known for its proprietary data transmission protocol, ensuring reliable connectivity, and its battery lasts up to 10 hours, making it suitable for long-term monitoring and research studies. The BCI system used by \cite{pinho2014wireless} employs active, dry electrodes and supports up to 32 channels with a sampling rate range of 250 Hz to 1 kHz. It also incorporates the ADS1299 AFE for analog-to-digital conversion, ensuring high fidelity in signal capture. Data is transferred via Wi-Fi 802.11 b/g/n, enabling flexible and high-speed communication with external devices. The system boasts an impressive 26-hour battery life, making it an excellent option for extended usage in field studies or clinical applications. The BCI system described by \cite{Totev2023} uses passive, dry electrodes and supports up to 40 channels with a sampling rate of 250 Hz. It incorporates the ADS1298 AFE for high-quality data acquisition and utilizes RF (Radio Frequency) for data transfer. While battery life details are not specified, this device is likely designed for portable, research-focused applications where wireless data transfer is essential for real-time monitoring. Finally, the \cite{Beats} system features 32 flexible, wet electrodes and uses the ADS1299 AFE for high-precision EEG signal acquisition at a sampling rate of 4 kHz. Data is transmitted wirelessly via Wi-Fi, allowing for real-time data monitoring and analysis. The system’s battery life is 24 hours when wired, providing extended operation for intensive studies or clinical assessments that require continuous monitoring.

Each of these devices represents a different approach to EEG signal acquisition, offering varying numbers of channels, electrode types, sampling rates, and battery life. While some are optimized for research and clinical use with high sampling rates and extended battery life, others are more suited to consumer applications with lower sampling rates and shorter operational times. The choice of device depends largely on the specific needs of the user, whether for research, clinical monitoring, or personal use in neurofeedback and cognitive training applications.




%\subsection{Evolution of Portable High-Fidelity EEG Architectures}

%\subsection{Synchronization Paradigms: The "Jitter" Problem in Neurogaming}