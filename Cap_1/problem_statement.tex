\section{Problem statement}
\label{sec:problem} 

The design and implementation of serious games synchronized with neurophysiological signals such as electroencephalography (EEG) presents a critical challenge, especially when targeting cognitive stimulation and diagnostic support in pediatric populations with Attention Deficit Hyperactivity Disorder (ADHD)~\cite{raza2025effectiveness}. The scientific validity of such applications is fundamentally dependent on the precise temporal synchronization of at least two disparate data streams: the high-temporal-resolution physiological data from the EEG system and the context-dependent event data generated by the serious game~\cite{romani2025brainform}. A core technical obstacle lies in achieving this precise synchronization, a requirement that is essential for both the accuracy of event-related potential (ERP) measurements and the effectiveness of real-time interventions~\cite{fanourakis2024amucs}. This thesis addresses two primary facets of this challenge: the temporal inaccuracies introduced by system-level operations and the physical limitations of the hardware itself.

\subsection{Unpredictable Latency and Jitter Between Game Events and EEG Recordings}

The primary technical issue in synchronizing EEG data with serious game events is the presence of unpredictable latency and jitter. Latency is the delay between an event's physical occurrence (e.g., a stimulus appearing on screen) and its corresponding timestamp being recorded in the data stream. A more pernicious issue is jitter, defined as the statistical variability in that latency over time. While a constant latency might be correctable in post-processing, jitter introduces random, unpredictable timing errors that cannot be easily removed after data acquisition~\cite{iwama2023two}.

This issue is caused by several interrelated factors: buffering delays in data pipelines, the non-deterministic scheduling of non-real-time operating systems, variability in communication protocols (such as USB, Bluetooth, or the Lab Streaming Layer), and asynchronous execution within game engines like Unity. These conditions lead to a lack of temporal precision, where the timestamp of an in-game event does not accurately align with the corresponding entry in the EEG data stream~\cite{cardona2023novel}.

This misalignment significantly compromises the quality of neurophysiological analysis. ERP components such as the P300 and N200, which are commonly used to evaluate attentional processes in ADHD, depend on millisecond-level synchronization between stimulus onset and neural response. When event markers are not precisely aligned due to jitter, the resulting ERP waveform becomes temporally "smeared," causing a reduction in both amplitude and interpretability, which degrades the signal-to-noise ratio and threatens diagnostic reliability~\cite{klee2024effect}. This is particularly critical in pediatric populations where subtle attentional deficits are being assessed.

Furthermore, in real-time systems like neurofeedback applications, where immediate feedback is essential for operant conditioning, even minor delays can disrupt the feedback loop. If the user receives auditory or visual feedback that no longer corresponds precisely to their brain state, the therapeutic effectiveness is reduced, potentially leading to user disengagement or ineffective training outcomes. The temporal precision required is demanding; some brain-computer interface (BCI) paradigms require accuracy within $\pm$2 milliseconds, yet jitter introduced by a game's graphical rendering at 50 frames per second can be as high as 20 milliseconds~\cite{wang2025review}.

Recent studies have quantified these challenges. For example,~\cite{larsen2024method} found that even in systems optimized with Unity and LSL, event marker delays averaged 36 milliseconds with a jitter of 5 to 6 milliseconds—well above the acceptable margin for many ERP analyses. Additionally, Brain Products~\cite{BrainProducts2025} reports that embedded platforms lacking efficient buffering and timestamping can exhibit latencies up to 100 milliseconds, particularly under high computational load. These delays, caused by a lack of dedicated real-time scheduling and protocol optimization, result in a substantial loss of synchronization fidelity, ultimately undermining both research validity and clinical utility.

\subsection{Resource and power constraints in embedded EEG platforms}

The second major issue stems from the computational and energy limitations of embedded and wearable EEG systems. Designed to be mobile and unobtrusive, these systems often operate on limited battery power, constrained CPU cycles, and reduced memory~\cite{kaongoen2023future}. These constraints are exacerbated when the system must simultaneously support real-time data acquisition, multichannel EEG streaming, and high-frequency event marker registration. Conventional EEG setups that rely on centralized data processing can also lead to high energy consumption and increased data transmission latency~\cite{zhang2025recent, akhtar2025ai}.
These limitations make it difficult to implement low-latency communication and high-resolution timestamping. Wireless data transmission, in particular, is very power-intensive~\cite{kumar2022fpga}. Protocols such as Bluetooth and Wi-Fi—commonly used in portable EEG systems—can introduce packet retransmissions, buffering delays, and inconsistent delivery times that worsen synchronization accuracy~\cite{lin2023design}.

The consequences are significant. System designers are forced to lower EEG sampling rates, simplify marker handling, or accept increased delays—all of which reduce the reliability of the collected data and the interactivity of the game~\cite{keutayeva2025neurotechnology}. For example, a review of wearable EEG systems found that wireless devices consistently showed worse timing stability and synchronization performance compared to wired configurations, especially when embedded resources were under heavy computational load~\cite{barbera2025using}.

Brain Products~\cite{BrainProducts2025} corroborates these findings, warning that system performance degrades as channel count and sampling rate increase—conditions commonly required in clinical-grade EEG systems. This creates a fundamental trade-off: increasing signal fidelity and temporal resolution compromises system portability, while optimizing for mobility sacrifices diagnostic precision.

Therefore, a critical gap exists in establishing a robust methodology to reliably synchronize multimodal data streams from EEG systems and dynamic serious games with quantifiable, millisecond-level precision, while also operating within the power and resource constraints of embedded platforms. This thesis addresses the problem of ensuring the temporal integrity of these data streams to enable scientifically valid analysis of neuro-cognitive processes during gameplay.




% \begin{figure}
%     \centering
%     \includegraphics[width=0.8\linewidth]{Cap_1/Figures/state of the art.png}
%     \caption{Challenges in EEG systems and serious games for ADHD evaluation.}
%     \label{fig:Esquema}
% \end{figure}




\newpage
\section{Research question}\label{sec:question}

How can a low-latency and low-jitter data synchronization framework be developed and validated to ensure the temporal integrity of multimodal data from embedded EEG systems and dynamic serious game events, while respecting the inherent resource and power constraints of such platforms?


