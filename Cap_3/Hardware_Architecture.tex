\chapter{Hardware Architecture (The MONEEE System)}
\label{ch:hardware_architecture}

This chapter details the engineering design of the MONEEE system. The architecture is designed to address the specific requirement of low-latency synchronization between game events and physiological signals. The system is built upon a heterogeneous embedded platform that physically separates the real-time acquisition domain from the high-level computational domain.

\section{System Overview \& Block Diagram}
\label{sec:system_overview}

The MONEEE system operates as a dedicated edge-computing device for EEG acquisition. The data flow is strictly unidirectional for physiological signals, designed to minimize latency and maximizing signal integrity. The high-level signal chain is defined as follows:

\begin{equation}
    \text{Electrodes} \xrightarrow{\text{Analog}} \text{ADS1299} \xrightarrow{\text{SPI}} \text{TM4C1294} \xrightarrow{\text{UART/SPI}} \text{RPi CM4}
\end{equation}

The architecture is divided into three primary zones: the \textit{Analog Front-End}, the \textit{Real-Time Core}, and the \textit{Compute Core}.

\begin{figure}[h]
    \centering
    % \includegraphics[width=1.0\textwidth]{moneee_block_diagram.png}
    \caption{High-level block diagram of the MONEEE system showing the separation between the acquisition (MCU) and compute (MPU) domains.}
    \label{fig:block_diagram}
\end{figure}

\section{Analog Front-End (AFE)}
\label{sec:analog_front_end}

The Analog Front-End is the interface between the biological medium and the digital system. It is built around the Texas Instruments ADS1299, an 8-channel, 24-bit delta-sigma ADC designed specifically for biopotential measurements.

\subsection{Configuration of the ADS1299}
The ADS1299 is configured to operate in a low-noise, high-gain mode suitable for scalp EEG. The key configuration parameters implemented in the register settings are:
\begin{itemize}
    \item \textbf{Gain:} Programmable Gain Amplifier (PGA) set to $24V/V$ to maximize the dynamic range for small EEG signals ($10-100 \mu V$).
    \item \textbf{Data Rate:} Configured for 250 SPS (Samples Per Second) or 500 SPS, providing a bandwidth sufficient for capturing the P300 and N200 components (typically $<30$ Hz) while allowing for oversampling benefits.
    \item \textbf{Input Multiplexer:} Set to \texttt{NORMAL} electrode input, with options to switch to internal test signals for calibration.
\end{itemize}

\subsection{Bias Drive Implementation (Driven Right Leg)}
To reject common-mode noise (such as 50/60 Hz mains interference), the system utilizes a Driven Right Leg (DRL) circuit, referred to as the \textit{Bias Drive} in the ADS1299 architecture.
Instead of a passive ground, the DRL circuit measures the common-mode voltage on the sensing electrodes, inverts it, amplifies it, and drives it back into the body through a reference electrode. This negative feedback loop actively cancels interference, significantly improving the Common Mode Rejection Ratio (CMRR) to typically $>110$ dB.

\subsection{Power Supply Isolation}
Safety and noise performance dictate the power architecture. The AFE is powered by a dedicated Li-Po battery managed by a PMIC (Power Management IC). Crucially, the analog power domain ($AVDD$) is isolated from the noisy digital domains of the Raspberry Pi using Low-Dropout Regulators (LDOs) with high Power Supply Rejection Ratio (PSRR). This ensures that the high-frequency switching noise from the Compute Module's CPU rails does not couple into the sensitive EEG measurements.

\section{The Digital Core (The Split-Architecture)}
\label{sec:digital_core}

The digital processing load is distributed between a microcontroller and a microprocessor, leveraging the strengths of each.

\subsection{Real-Time Unit (TI TM4C1294)}
The Texas Instruments TM4C1294 (ARM Cortex-M4F) serves as the hard real-time controller.
\begin{itemize}
    \item \textbf{Role:} It acts as the SPI Master for the ADS1299. Its primary responsibility is to service the \texttt{DRDY} (Data Ready) interrupt from the ADC immediately upon assertion, ensuring zero sample loss.
    \item \textbf{Floating-Point Unit (FPU):} The hardware FPU allows for real-time application of basic digital filters (e.g., notch filters for line noise) or scaling factors before data transmission, without stalling the interrupt service routines.
    \item \textbf{Determinism:} Unlike the Linux-based CM4, the TM4C code runs on bare metal (or a lightweight RTOS), guaranteeing that the timestamp applied to each incoming data packet is accurate to within microseconds.
\end{itemize}

\subsection{Compute Unit (Raspberry Pi CM4)}
The Raspberry Pi Compute Module 4 acts as the system's "Host."
\begin{itemize}
    \item \textbf{Role:} It manages high-bandwidth data storage (to eMMC or SD card), runs the Lab Streaming Layer (LSL) gateway, and handles network transmission via Wi-Fi.
    \item \textbf{Throughput:} The CM4 processes the incoming stream from the TM4C, formats it into LSL chunks, and broadcasts it to the serious game running on the tablet or local network.
\end{itemize}

\subsection{Inter-Processor Communication}
Data is transferred from the TM4C1294 to the RPi CM4 via a high-speed serial interface.
\begin{itemize}
    \item \textbf{Physical Layer:} A high-speed UART connection (baud rate $>921600$) or SPI is utilized. To prevent ground loops between the battery-powered AFE and the potentially mains-connected tablet (if charging), this link is galvanically isolated using digital isolators (e.g., ISO77xx series).
    \item \textbf{Protocol:} A lightweight binary packet protocol is defined, wrapping the 24-bit EEG data and the 32-bit hardware timestamp into a frame with a cyclic redundancy check (CRC) to ensure data integrity during transmission.
\end{itemize}

\section{The Event Interface (USB-C)}
\label{sec:event_interface}

The physical interface for the serious game tablet is a USB Type-C connector.

\subsection{Hardware Implementation}
The USB-C port is configured as a downstream facing port (or device mode depending on the tablet role) using a USB Controller integrated into the TM4C or CM4 carrier board. This port handles the reception of "event markers" sent by the game.

\subsection{Signal Conditioning and Isolation}
Connecting a commercial tablet via USB introduces significant noise risks. The tablet's internal DC-DC converters can inject noise into the USB ground line.
To mitigate this, the MONEEE system employs full USB isolation. The data lines ($D+/D-$) pass through a specialized USB isolator IC (e.g., ADuM3160 or similar), effectively breaking the galvanic path. This ensures that the "clean" ground of the EEG sensors remains floating relative to the "dirty" ground of the tablet, preserving the signal-to-noise ratio required for detecting the ERPs.