\newpage
\chapter*{\sffamily Resumen}
\addcontentsline{toc}{chapter}{Resumen}%
\par 

 

\textbf{Palabras clave:} TDAH, sincronización EEG, interfaces cerebro-computadora, juegos serios, baja latencia, adquisición de señales, entornos educativos, sistemas portátiles


\newpage 
\chapter*{\sffamily Abstract}
\addcontentsline{toc}{chapter}{Abstract}%
\par 

Attention Deficit Hyperactivity Disorder (ADHD) is a neuropsychiatric condition that affects approximately 10\% of children in Colombia. Current diagnostic methods rely heavily on subjective symptom assessments, increasing the risk of overdiagnosis and overtreatment. Emerging technologies, such as Brain-Computer Interfaces (BCIs) and serious games, offer a promising alternative by enabling real-time capture of EEG signals, thus providing a more objective assessment of the cognitive and emotional patterns associated with ADHD.

A key technical challenge in these applications is achieving precise synchronization between serious game events and EEG signals, as temporal misalignment compromises the validity of time-based analyses. This issue is exacerbated by factors such as data transmission latencies and the low channel density typical of portable EEG devices. Furthermore, EEG signals are highly susceptible to noise and physiological artifacts, which affect their reliability. Wireless transmission protocols introduce significant latency, particularly in systems designed to be accessible and cost-effective, limiting precision in clinical and educational applications.

This work presents an optimized architecture for EEG signal acquisition, focused on minimizing transmission latency and improving temporal synchronization. The solution implements low-latency transmission algorithms, robust alignment techniques, and spatial resolution enhancement strategies through a high-density channel system. The proposed approach is designed for applications in clinical and educational environments, contributing to more accurate and effective assessment and intervention for children with ADHD.

\textbf{Keywords:} ADHD, EEG synchronization, brain-computer interfaces, serious games, low latency, signal acquisition, educational environments, portable systems 




%\newpage 
%\chapter*{\sffamily Zusammenfassung}
%\addcontentsline{toc}{chapter}{Zusammenfassung}%
%\par Zusammenfassung texte.
%\par 
%\\[2cm]
%\textbf{Schlüsselwörter:} \schlusselworter