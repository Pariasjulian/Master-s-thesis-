\section{Estado del arte}\label{sec:stateofart}


In recent years, numerous wireless systems for EEG data acquisition have been developed, with two main approaches standing out: conventional remote monitoring systems and portable smart systems. The former simply digitize the EEG signals and transmit them to a remote unit for processing, usually in a deferred manner \cite{arpaia2020wearable}. On the other hand, portable systems preprocess the signals on a local device, such as a microcontroller (MCU), and wirelessly transmit the data using low-power consumption protocols. \ref{table:bci_hardware} This latter approach is crucial for real-time applications, where low latency is essential.

Marker synchronization in portable EEG acquisition systems \cite{razavi2022opensync}, particularly in applications combined with serious games \cite{damavsevivcius2023serious}, faces several technical and operational challenges. One of the main issues lies in latency in data transmission protocols. In portable EEG systems, precise synchronization between brain events and interactions in the game is crucial \cite{gomezromero2024implications}, but inherent limitations of portable acquisition systems, such as latencies in data transfer protocols, can cause temporal mismatches. These latencies primarily stem from bandwidth constraints in wireless transmission and the need to process large volumes of data in real-time \cite{he2023diversity}.

The type of electrode \cite{liu2023feature} and the number of channels \cite{abdullah2022eeg}  are determining factors in the quality of the data acquired in portable systems. Although dry electrodes offer greater portability, they tend to generate lower-quality signals due to reduced conductivity, which can complicate precise synchronization with other devices, such as serious games. On the other hand, the use of systems with \textbf{low channel density (e.g., 8-16 channels)} \cite{allouch2023effect} is a common strategy in these portable systems to minimize size and improve portability. However, low channel density can affect the spatial resolution of EEG data, limiting the ability to perform accurate analysis of brain patterns. This challenge is reflected in the need to optimize sampling \cite{zheng2023effects}  and data transfer protocols \cite{bayilmics2022survey}  to ensure that captured signals are transmitted efficiently without significant information loss

The sampling rate is another critical factor, as it directly affects the temporal resolution of EEG signals. The combination of low channel density and insufficient sampling rate can make it difficult to capture fast brain events, such as attention shifts, which are essential in applications like serious games. Furthermore, Signal Front-End Amplifiers (AFE) \cite{devi2022survey} play a key role in signal quality. While low-cost AFEs may be suitable for portable systems, they tend to have limitations in processing capacity, which impacts data synchronization by generating noise and distortions in the EEG signals, especially when connected to mobile devices with lower processing power.

Battery life \cite{niso2023wireless} is a significant constraint for portable systems that require long monitoring sessions. EEG systems that operate for several hours often need to optimize their energy consumption, which may involve reducing the sampling rate or channel density, once again impacting data quality and real-time synchronization.


\begin{table}[H]
    \caption{Dispositivos de adquisición utilizados para BCI. La tabla ofrece una descripción general de los diferentes dispositivos de hardware, sus especificaciones y protocolos de comunicación.}
    \label{table:bci_hardware}
    \centering
    \begin{tabular}{|p{2cm}|p{1.5cm}|p{1.6cm}|p{1.6cm}|p{1.7cm}|p{1.4cm}|p{1.8cm}|p{1.4cm}|}
        \hline
        \textbf{Hardware BCI} & \textbf{Empresa} & \textbf{Tipo de Electrodo} & \textbf{Canales} & \textbf{Frecuencia de Muestreo} & \textbf{AFE} & \textbf{Protocolo y Transferencia de Datos} & \textbf{Duración de la Batería} \\ \hline
        Cyton + Daisy \cite{OpenBCI_CytonDaisy} & OpenBCI & Flexible / Húmedo / Seco & 16 & 250 Hz- -16 kHz & ADS1299 & RF / BLE / Wi-Fi & 8 h \\ \hline
        actiCAP \cite{BrainProducts_ActiCap} & Brain Products GmbH & Flexible / Húmedo / Seco & 16 & 256 Hz- -16 kHz & - & USB & 16 h \\ \hline
        EPOC X \cite{Emotiv_EPOCX} & Emotiv & Rígido / Húmedo & 14 & 128 Hz & - & BLE / Bluetooth & 6--12 h \\ \hline
        Diadem \cite{Bitbrain_Diadem} & Bitbrain & Rígido / Seco & 12 & 256 Hz & - & Bluetooth & 8 h \\ \hline
        g.Nautilus \cite{Gtec_GNautilusProFlexible} & g.tec & Flexible & 8 / 16 / 32 & 250 Hz & ADS1299 & Propietario & 10 h \\ \hline
        Plataforma para EEG ambulatorio \cite{pinho2014wireless} & - & Activo / Seco & 32 & 250 Hz- -1 kHz & ADS1299 & Wi-Fi 802.11 b/g/n & 26 h \\ \hline
        Sistema para neurofeedback \cite{Totev2023} & - & Pasivo / Seco & 40 & 250 Hz & ADS1298 & RF & - \\ \hline
        BEATS \cite{Beats} & - & Flexible / Húmedo & 32 & 4 kHz & ADS1299 & Wi-Fi & 24 h (cableado) \\ \hline
    \end{tabular}
\end{table}


% \begin{table}[H]
%     \caption{Acquisition devices used for BCI. The table provides an overview of different hardware devices, their specifications, and communication protocols.}
%     \label{table:bci_hardware}
    
% \begin{adjustwidth}{-\extralength}{0cm}
%         \setlength{\cellWidtha}{\fulllength/8-2\tabcolsep+0.6in}
%         \setlength{\cellWidthb}{\fulllength/8-2\tabcolsep-0.1in}
%         \setlength{\cellWidthc}{\fulllength/8-2\tabcolsep-0in}
%         \setlength{\cellWidthd}{\fulllength/8-2\tabcolsep-0in}
%         \setlength{\cellWidthe}{\fulllength/8-2\tabcolsep-0.2in}
%         \setlength{\cellWidthf}{\fulllength/8-2\tabcolsep-0.1in}
%         \setlength{\cellWidthg}{\fulllength/8-2\tabcolsep-0.1in}
%         \setlength{\cellWidthh}{\fulllength/8-2\tabcolsep-0.1in}
%         \scalebox{1}[1]{\begin{tabularx}{\fulllength}{
%                     >{\PreserveBackslash\centering}m{\cellWidtha}
%                     >{\PreserveBackslash\centering}m{\cellWidthb}
%                     >{\PreserveBackslash\centering}m{\cellWidthc}
%                     >{\PreserveBackslash\centering}m{\cellWidthd}
%                     >{\PreserveBackslash\centering}m{\cellWidthe}
%                     >{\PreserveBackslash\centering}m{\cellWidthf}
%                     >{\PreserveBackslash\centering}m{\cellWidthg}
%                     >{\PreserveBackslash\centering}m{\cellWidthh}}
%                 \toprule 
%                 \textbf{BCI Hardware} & \textbf{Company} & \textbf{Electrode Type} & \textbf{Channels} & \textbf{Sampling Rate} & \textbf{AFE} & \textbf{Protocol and Data Transfer} & \textbf{Battery Life} \\
%                 \midrule
%                 Cyton + Daisy \cite{OpenBCI_CytonDaisy} & OpenBCI & Flexible/Wet/Dry & 16 & 250 Hz--16 kHz & ADS1299 & RF/BLE/Wi-Fi & 8 h \\
%                 actiCAP\cite{BrainProducts_ActiCap} & Brain Products GmbH & Flexible/Wet/Dry & 16 & 256 Hz--16 kHz & - & USB & 16 h \\
                
%                 EPOC X \cite{Emotiv_EPOCX}& Emotiv & Rigid/Wet & 14 & 128 Hz & - & BLE/Bluetooth & 6--12 h \\
                
%                 Diadem \cite{Bitbrain_Diadem}& Bitbrain & Rigid/Dry & 12 & 256 Hz & - & Bluetooth & 8 h \\
                
%                 g.Nautilus \cite{Gtec_GNautilusProFlexible} & g.tec & Flexible & 8/16/32 & 250 Hz & ADS1299 & Proprietary & 10 h \\
                
%                 Platform for ambulatory EEG \cite{pinho2014wireless} & - & Active/Dry & 32 & 250 Hz--1 kHz & ADS1299 & Wi-Fi 802.11 b/g/n & 26 h \\
                
%                 System for neurofeedback \cite{Totev2023} & - & Passive/Dry & 40 & 250 Hz & ADS1298 & RF & - \\
                
%                 BEATS \cite{Beats} & - & Flexible/Wet & 32 & 4 kHz & ADS1299 & Wi-Fi & 24 h (wired) \\
%                 \bottomrule
%         \end{tabularx}}
%     \end{adjustwidth}
% \end{table}

In the field of brain-computer interfaces (BCIs), several devices have been developed, each with unique features tailored to specific use cases such as clinical research, neurofeedback, or consumer applications. The Cyton + Daisy system by OpenBCI \cite{OpenBCI_CytonDaisy} supports up to 16 channels and offers a wide sampling rate range of 250 Hz to 16 kHz, making it suitable for high-resolution EEG acquisition. The device uses flexible, wet, or dry electrodes and incorporates the ADS1299 AFE for high-quality signal conversion. It supports data transfer via RF, Bluetooth Low Energy (BLE), and Wi-Fi, allowing for versatile connectivity. With a battery life of 8 hours, this system is highly adaptable, suitable for both research and practical applications in various environments. Another system, actiCAP \cite{BrainProducts_ActiCap} by Brain Products GmbH, features flexible, wet, or dry electrodes and is capable of recording up to 16 channels with a sampling rate range from 256 Hz to 16 kHz. The actiCAP does not use a dedicated AFE and instead relies on a USB protocol for data transfer. The device provides a robust 16-hour battery life, making it an ideal choice for long-duration experiments and clinical settings that require stable signal acquisition over extended periods. The EPOC X \cite{Emotiv_EPOCX} by Emotiv is a more compact and consumer-oriented BCI device that uses rigid, wet electrodes and supports 14 channels with a sampling rate of 128 Hz. This device employs Bluetooth Low Energy (BLE) for wireless data transfer, and its battery life ranges from 6 to 12 hours, depending on usage. While its lower sampling rate may limit its use for high-resolution research, the EPOC X remains a popular choice for applications in neurofeedback, cognitive training, and general user interaction. The Diadem \cite{Bitbrain_Diadem} system by Bitbrain uses rigid, dry electrodes and supports 12 channels with a sampling rate of 256 Hz. It operates via Bluetooth for data transmission and has a battery life of 8 hours, providing a balance between portability and signal quality. The g.Nautilus \cite{Gtec_GNautilusProFlexible} system by g.tec offers great flexibility, supporting configurations with 8, 16, or 32 channels. It operates at a sampling rate of 250 Hz and uses the ADS1299 AFE for high-performance signal acquisition. The system is known for its proprietary data transmission protocol, ensuring reliable connectivity, and its battery lasts up to 10 hours, making it suitable for long-term monitoring and research studies. The BCI system used by \cite{pinho2014wireless} employs active, dry electrodes and supports up to 32 channels with a sampling rate range of 250 Hz to 1 kHz. It also incorporates the ADS1299 AFE for analog-to-digital conversion, ensuring high fidelity in signal capture. Data is transferred via Wi-Fi 802.11 b/g/n, enabling flexible and high-speed communication with external devices. The system boasts an impressive 26-hour battery life, making it an excellent option for extended usage in field studies or clinical applications. The BCI system described by \cite{Totev2023} uses passive, dry electrodes and supports up to 40 channels with a sampling rate of 250 Hz. It incorporates the ADS1298 AFE for high-quality data acquisition and utilizes RF (Radio Frequency) for data transfer. While battery life details are not specified, this device is likely designed for portable, research-focused applications where wireless data transfer is essential for real-time monitoring. Finally, the \cite{Beats} system features 32 flexible, wet electrodes and uses the ADS1299 AFE for high-precision EEG signal acquisition at a sampling rate of 4 kHz. Data is transmitted wirelessly via Wi-Fi, allowing for real-time data monitoring and analysis. The system’s battery life is 24 hours when wired, providing extended operation for intensive studies or clinical assessments that require continuous monitoring.

Each of these devices represents a different approach to EEG signal acquisition, offering varying numbers of channels, electrode types, sampling rates, and battery life. While some are optimized for research and clinical use with high sampling rates and extended battery life, others are more suited to consumer applications with lower sampling rates and shorter operational times. The choice of device depends largely on the specific needs of the user, whether for research, clinical monitoring, or personal use in neurofeedback and cognitive training applications.


% En los últimos años, se han desarrollado numerosos sistemas inalámbricos para la adquisición de datos EEG, destacándose dos enfoques principales: los sistemas convencionales de monitoreo remoto y los sistemas portátiles inteligentes. Los primeros simplemente digitalizan las señales EEG y las transmiten a una unidad remota para su procesamiento, generalmente de manera diferida \cite{arpaia2020wearable}. Por otro lado, los sistemas portátiles preprocesan las señales en un dispositivo local, como un microcontrolador (MCU), y transmiten los datos de manera inalámbrica utilizando protocolos de bajo consumo energético. \ref{table:bci_hardware} Este último enfoque es crucial para aplicaciones en tiempo real, donde la baja latencia es esencial.

% La duración de la batería \cite{niso2023wireless} es una restricción significativa para los sistemas portátiles que requieren sesiones de monitoreo largas. Los sistemas EEG que operan durante varias horas a menudo necesitan optimizar su consumo de energía, lo que puede implicar la reducción de la tasa de muestreo o la densidad de canales, lo que nuevamente impacta la calidad de los datos y la sincronización en tiempo real.

% La sincronización de marcadores en sistemas portátiles de adquisición de EEG \cite{razavi2022opensync}, particularmente en aplicaciones combinadas con juegos serios \cite{damavsevivcius2023serious}, enfrenta varios desafíos técnicos y operativos. Uno de los principales problemas radica en la latencia de los protocolos de transmisión de datos. En los sistemas portátiles de EEG, la sincronización precisa entre los eventos cerebrales y las interacciones en el juego es crucial \cite{gomezromero2024implications}, pero las limitaciones inherentes de los sistemas de adquisición portátiles, como las latencias en los protocolos de transferencia de datos, pueden causar desajustes temporales. Estas latencias provienen principalmente de las limitaciones de ancho de banda en la transmisión inalámbrica y la necesidad de procesar grandes volúmenes de datos en tiempo real \cite{he2023diversity}.

% El tipo de electrodo \cite{liu2023feature} y el número de canales \cite{abdullah2022eeg} son factores determinantes en la calidad de los datos adquiridos en los sistemas portátiles. Aunque los electrodos secos ofrecen mayor portabilidad, tienden a generar señales de menor calidad debido a la conductividad reducida, lo que puede complicar la sincronización precisa con otros dispositivos, como los juegos serios. Por otro lado, el uso de sistemas con \textbf{baja densidad de canales (por ejemplo, 8-16 canales)} \cite{allouch2023effect} es una estrategia común en estos sistemas portátiles para minimizar el tamaño y mejorar la portabilidad. Sin embargo, la baja densidad de canales puede afectar la resolución espacial de los datos EEG, limitando la capacidad para realizar un análisis preciso de los patrones cerebrales. Este desafío se refleja en la necesidad de optimizar la tasa de muestreo \cite{zheng2023effects} y los protocolos de transferencia de datos \cite{bayilmics2022survey} para asegurar que las señales capturadas se transmitan de manera eficiente sin pérdida significativa de información.

% La tasa de muestreo es otro factor crítico, ya que afecta directamente la resolución temporal de las señales EEG. La combinación de baja densidad de canales e insuficiente tasa de muestreo puede dificultar la captura de eventos cerebrales rápidos, como los cambios de atención, que son esenciales en aplicaciones como los juegos serios. Además, los amplificadores de señal front-end (AFE) \cite{devi2022survey} juegan un papel clave en la calidad de la señal. Si bien los AFE de bajo costo pueden ser adecuados para sistemas portátiles, tienden a tener limitaciones en capacidad de procesamiento, lo que impacta la sincronización de los datos al generar ruido y distorsiones en las señales EEG, especialmente cuando están conectados a dispositivos móviles con menor poder de procesamiento.  

% La duración de la batería \cite{niso2023wireless} es una restricción significativa para los sistemas portátiles que requieren sesiones de monitoreo largas. Los sistemas EEG que operan durante varias horas a menudo necesitan optimizar su consumo de energía, lo que puede implicar la reducción de la tasa de muestreo o la densidad de canales, lo que nuevamente impacta la calidad de los datos y la sincronización en tiempo real.

% \begin{table}[H]
%     \caption{Dispositivos de adquisición utilizados para BCI. La tabla ofrece una descripción general de los diferentes dispositivos de hardware, sus especificaciones y protocolos de comunicación.}
%     \label{table:bci_hardware}
%     \centering
%     \begin{tabular}{|p{2cm}|p{1.5cm}|p{1.6cm}|p{1.6cm}|p{1.7cm}|p{1.4cm}|p{1.8cm}|p{1.4cm}|}
%         \hline
%         \textbf{Hardware BCI} & \textbf{Empresa} & \textbf{Tipo de Electrodo} & \textbf{Canales} & \textbf{Frecuencia de Muestreo} & \textbf{AFE} & \textbf{Protocolo y Transferencia de Datos} & \textbf{Duración de la Batería} \\ \hline
%         Cyton + Daisy \cite{OpenBCI_CytonDaisy} & OpenBCI & Flexible / Húmedo / Seco & 16 & 250 Hz- -16 kHz & ADS1299 & RF / BLE / Wi-Fi & 8 h \\ \hline
%         actiCAP \cite{BrainProducts_ActiCap} & Brain Products GmbH & Flexible / Húmedo / Seco & 16 & 256 Hz- -16 kHz & - & USB & 16 h \\ \hline
%         EPOC X \cite{Emotiv_EPOCX} & Emotiv & Rígido / Húmedo & 14 & 128 Hz & - & BLE / Bluetooth & 6--12 h \\ \hline
%         Diadem \cite{Bitbrain_Diadem} & Bitbrain & Rígido / Seco & 12 & 256 Hz & - & Bluetooth & 8 h \\ \hline
%         g.Nautilus \cite{Gtec_GNautilusProFlexible} & g.tec & Flexible & 8 / 16 / 32 & 250 Hz & ADS1299 & Propietario & 10 h \\ \hline
%         Plataforma para EEG ambulatorio \cite{pinho2014wireless} & - & Activo / Seco & 32 & 250 Hz- -1 kHz & ADS1299 & Wi-Fi 802.11 b/g/n & 26 h \\ \hline
%         Sistema para neurofeedback \cite{Totev2023} & - & Pasivo / Seco & 40 & 250 Hz & ADS1298 & RF & - \\ \hline
%         BEATS \cite{Beats} & - & Flexible / Húmedo & 32 & 4 kHz & ADS1299 & Wi-Fi & 24 h (cableado) \\ \hline
%     \end{tabular}
% \end{table}

% En el campo de las interfaces cerebro-computadora (BCIs), se han desarrollado varios dispositivos, cada uno con características únicas adaptadas a casos de uso específicos como la investigación clínica, el neurofeedback o las aplicaciones para consumidores. El sistema Cyton + Daisy de OpenBCI \cite{OpenBCI_CytonDaisy} soporta hasta 16 canales y ofrece un amplio rango de tasa de muestreo de 250 Hz a 16 kHz, lo que lo hace adecuado para la adquisición de EEG de alta resolución. El dispositivo utiliza electrodos flexibles, húmedos o secos e incorpora el AFE ADS1299 para la conversión de señales de alta calidad. Soporta la transferencia de datos mediante RF, Bluetooth Low Energy (BLE) y Wi-Fi, lo que permite una conectividad versátil. Con una duración de batería de 8 horas, este sistema es altamente adaptable, adecuado tanto para la investigación como para aplicaciones prácticas en diversos entornos. Otro sistema, el actiCAP \cite{BrainProducts_ActiCap} de Brain Products GmbH, presenta electrodos flexibles, húmedos o secos y es capaz de grabar hasta 16 canales con un rango de tasa de muestreo de 256 Hz a 16 kHz. El actiCAP no utiliza un AFE dedicado, sino que depende de un protocolo USB para la transferencia de datos. El dispositivo proporciona una robusta duración de batería de 16 horas, lo que lo convierte en una opción ideal para experimentos de larga duración y entornos clínicos que requieren adquisición de señales estable a lo largo de períodos prolongados. El EPOC X \cite{Emotiv_EPOCX} de Emotiv es un dispositivo BCI más compacto y orientado al consumidor que utiliza electrodos rígidos y húmedos y soporta 14 canales con una tasa de muestreo de 128 Hz. Este dispositivo emplea Bluetooth Low Energy (BLE) para la transferencia de datos inalámbrica, y su duración de batería varía entre 6 y 12 horas, dependiendo del uso. Si bien su tasa de muestreo más baja puede limitar su uso para investigaciones de alta resolución, el EPOC X sigue siendo una opción popular para aplicaciones de neurofeedback, entrenamiento cognitivo e interacción general con el usuario. El sistema Diadem \cite{Bitbrain_Diadem} de Bitbrain utiliza electrodos rígidos y secos y soporta 12 canales con una tasa de muestreo de 256 Hz. Funciona mediante Bluetooth para la transmisión de datos y tiene una duración de batería de 8 horas, ofreciendo un equilibrio entre portabilidad y calidad de la señal. El sistema g.Nautilus \cite{Gtec_GNautilusProFlexible} de g.tec ofrece una gran flexibilidad, soportando configuraciones con 8, 16 o 32 canales. Opera a una tasa de muestreo de 250 Hz y utiliza el AFE ADS1299 para una adquisición de señales de alto rendimiento. El sistema es conocido por su protocolo de transmisión de datos propietario, que asegura una conectividad confiable, y su batería dura hasta 10 horas, lo que lo hace adecuado para monitoreo a largo plazo y estudios de investigación. El sistema BCI utilizado por \cite{pinho2014wireless} emplea electrodos activos y secos y soporta hasta 32 canales con un rango de tasa de muestreo de 250 Hz a 1 kHz. También incorpora el AFE ADS1299 para la conversión de señales analógicas a digitales, asegurando alta fidelidad en la captura de señales. Los datos se transfieren mediante Wi-Fi 802.11 b/g/n, lo que permite una comunicación flexible y de alta velocidad con dispositivos externos. El sistema presume una impresionante duración de batería de 26 horas, lo que lo convierte en una excelente opción para su uso prolongado en estudios de campo o aplicaciones clínicas. El sistema BCI descrito por \cite{Totev2023} utiliza electrodos pasivos y secos y soporta hasta 40 canales con una tasa de muestreo de 250 Hz. Incorpora el AFE ADS1298 para una adquisición de datos de alta calidad y utiliza RF (Radiofrecuencia) para la transferencia de datos. Aunque no se especifican detalles sobre la duración de la batería, este dispositivo está diseñado probablemente para aplicaciones portátiles y centradas en la investigación donde la transferencia inalámbrica de datos es esencial para el monitoreo en tiempo real. Finalmente, el sistema \cite{Beats} presenta 32 electrodos flexibles y húmedos y utiliza el AFE ADS1299 para una adquisición de señales EEG de alta precisión a una tasa de muestreo de 4 kHz. Los datos se transmiten de forma inalámbrica a través de Wi-Fi, lo que permite el monitoreo y análisis de datos en tiempo real. La duración de la batería del sistema es de 24 horas cuando está conectado por cable, lo que permite una operación extendida para estudios intensivos o evaluaciones clínicas que requieren monitoreo continuo.

% Cada uno de estos dispositivos representa un enfoque diferente para la adquisición de señales EEG, ofreciendo diversos números de canales, tipos de electrodos, tasas de muestreo y duración de batería. Mientras que algunos están optimizados para la investigación y el uso clínico con altas tasas de muestreo y larga duración de batería, otros son más adecuados para aplicaciones para consumidores con tasas de muestreo más bajas y tiempos de operación más cortos. La elección del dispositivo depende en gran medida de las necesidades específicas del usuario, ya sea para investigación, monitoreo clínico o uso personal en aplicaciones de neurofeedback y entrenamiento cognitivo.

