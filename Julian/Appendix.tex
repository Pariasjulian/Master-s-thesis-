\chapter{Appendix}


\section{MONELIB: BIBLIOTECA PARA COMUNICACIÓN SISTEMA EMBEBIDO-VIDEOJUEGO}

\label{apendice:monelib}

\textbf{Descripción}: La librería MoneLib permite establecer una comunicación eficiente entre videojuegos desarrollados en Unity y un sistema embebido externo. Su función principal es la captura en tiempo real de eventos de interacción dentro del videojuego, los cuales son codificados en formato hexadecimal y enviados vía USB para su posterior análisis y procesamiento en hardware embebido.


\begin{itemize}
    \item[\textbf{Versión}:] 1.0 (Mayo 2025)
    
    \item[\textbf{Componentes}:]
    
     \begin{itemize}
        \item \textbf{MoneLibrary.cs}: Clase que inicializa la librería nativa de Android en Unity y proporciona métodos para la gestión del driver y el envío de datos.
        \item \textbf{monelib-release.aar}: Librería nativa de Android responsable de la comunicación USB.
        \item \textbf{AndroidManifest.xml}: Archivo de configuración que define los permisos de USB y las actividades necesarias para la operación.
    \end{itemize}
\end{itemize}



\section{Requerimientos del Sistema}
\begin{itemize}
    \item[\textbf{Unity}:] Versión 6000.0.45f1 o superior.
    \item[\textbf{Sistema Operativo}:] Android 12 (Snow Cone) o superior.
    \item[\textbf{Hardware}:] Dispositivo Android con soporte para USB-C Host.
    \item[\textbf{Microcontrolador}:] Un microcontrolador que sea compatible con un periférico USB en modo \textit{device}.
\end{itemize}

\section*{Instalación y Configuración}
\begin{enumerate}
    \item \textbf{Importar Librería}: Colocar el archivo \texttt{monelib-release.aar} en el directorio \texttt{Assets/Plugins/Android} del proyecto en Unity.
    \item \textbf{Agregar Clase}: Ubicar el script \texttt{MoneLibrary.cs} en la carpeta de \texttt{Scripts} del proyecto.
    \item \textbf{Modificar el Manifest}: Incluir los siguientes permisos y características en el archivo \texttt{AndroidManifest.xml}:
    \begin{verbatim}
<uses-feature android:name="android.hardware.usb.host" />
<uses-permission android:name="android.permission.USB_PERMISSION" />
    \end{verbatim}
    \item \textbf{Inicializar el Plugin}: En el método \texttt{Start()} de un objeto principal (ej. \texttt{GameStartup.cs}), llamar a la función de inicialización: \\
    \texttt{MoneLibrary.InitializePlugin("com.beepro.monelib.PluginInstance");}
\end{enumerate}

\section*{Descripción de Funcionalidades}
\subsection*{Inicialización del Plugin}
\begin{itemize}
    \item[\textbf{Método}:] \texttt{InitializePlugin(string pluginName)}.
    \item[\textbf{Acción}:] Este método obtiene una referencia a la actividad actual de Unity, crea una instancia del plugin de Android y pasa el contexto de la actividad al plugin nativo para establecer la comunicación.
\end{itemize}

\subsection*{Envío de Datos por USB}
\begin{itemize}
    \item[\textbf{Método}:] \texttt{SendUsbData(sbyte a)}.
    \item[\textbf{Acción}:] Invoca el método nativo \texttt{sendUsbData} de la librería de Android para transmitir un evento, codificado como un entero \texttt{sbyte}.
\end{itemize}

\subsection*{Eventos Configurados}
La librería está preconfigurada para manejar los siguientes eventos de juego:

\begin{table}[h!]
\centering
\begin{tabular}{|l|c|c|}
\hline
\textbf{Evento} & \textbf{Código Decimal} & \textbf{Código Hexadecimal} \\ \hline
Jugador marca "O" & 0 & 0x00 \\ \hline
Jugador marca "X" & 1 & 0x01 \\ \hline
Reinicio del juego & -1 & 0xFF \\ \hline
\end{tabular}
\caption{Tabla de eventos configurados en MoneLib.}
\label{tab:eventos_monelib}
\end{table}

\section*{Ejemplo de Integración}
El siguiente fragmento de código ilustra cómo enviar eventos desde el videojuego:

\begin{lstlisting}[language=CSharp, caption={Ejemplo de implementación de envío de eventos.}, label={code:monelib_example}]
void OnCellClick(CellEnum content)
{
    if (content == CellEnum.Cross)
    {
        MoneLibrary.SendUsbData(1); // Evento para "X"
    }
    else if (content == CellEnum.Zero)
    {
        MoneLibrary.SendUsbData(0); // Evento para "O"
    }
}

void OnRestartButtonClick()
{
    MoneLibrary.SendUsbData(-1); // Evento de reinicio
}
\end{lstlisting}

\section*{Consideraciones de Seguridad}
\begin{itemize}
    \item Se debe verificar que la aplicación solicite los permisos USB correctamente cuando se conecte al sistema embebido.
    \item Para prevenir la sobresaturación del canal de comunicación, se recomienda mantener un intervalo mínimo de 1 milisegundo entre el envío de eventos consecutivos.
\end{itemize}