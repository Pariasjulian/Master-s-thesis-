\section{Materiales y Métodos}



El estudio fue diseñado para evaluar el rendimiento y las capacidades del sistema MindWave en un paradigma de juego serio orientado a la evaluación de procesos cognitivos mediante la adquisición de datos EEG. El hardware del sistema integra el convertidor analógico-digital (ADC) ADS1299 de Texas Instruments, que proporciona una adquisición de señales de alta resolución, y un microcontrolador capaz de encadenar múltiples ADCs para soportar hasta 64 canales EEG. Se utilizaron electrodos flexibles húmedos para registrar las señales electrofisiológicas, garantizando una recolección de datos confiable y al mismo tiempo asegurando la comodidad del participante durante sesiones de grabación extendidas. La transmisión de datos se implementó mediante un protocolo de comunicación serial, optimizado para minimizar la latencia y asegurar una transferencia de datos estable y en tiempo real a una computadora externa.

La plataforma de software desarrollada para el sistema permite la visualización en tiempo real, el almacenamiento y la gestión de los datos EEG. Esta plataforma también está integrada con un servicio de almacenamiento basado en la nube para el análisis posterior y la retención de datos a largo plazo. Además, el software facilita la presentación de estímulos visuales y auditivos a los participantes, asegurando una sincronización precisa entre las señales EEG y los eventos experimentales mediante registros con marcas de tiempo. Esta sincronización permite un análisis detallado de las respuestas cerebrales a eventos específicos del juego.