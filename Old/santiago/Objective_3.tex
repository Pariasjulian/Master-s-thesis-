%\chapter{\Objectivethreename}
\chapter{Dynamic Activation Function for Modeling Demand Flow Behavior in Gas Networks}
\label{ch:chapter_4}

\section{Results and Discussion}

\subsection{compliance with system restrictions }
The implementation of our proposal, which introduces a dynamic activation function to model the behavior of demand flows, does not generate significant alterations to existing restrictions. In general terms, its performance is comparable to that observed in Chapter 3, where both the node balance and the Weymouth equation show higher tolerances compared to conventional technique. However, our solution is distinguished by presenting more adjusted interquantile ranges and, in the case of the second restriction, the performance remains stable even with the increase in the number of nodes, a result not achieved by the traditional strategy. Both methodologies meet the compression ratio restriction, but it is in the cost aspect that a noticeable improvement is observed compared to the data presented in the previous chapter. The implementation of our proposal, as detailed in Equation 3.1, succeeds in reducing the variability of costs in our model. This results in a centralization of the median at zero, promotes a more symmetrical cost distribution and tends to offer more cost-effective solutions compared to the previous methodology. These results endorse the design and selection of our proposal for the dynamic activation function
\begin{figure}[H]
    \centering
    \includegraphics[width=1\linewidth]{Cap4/Figures/Modelr1.pdf}
    \caption{Compliance with Restrictions: The graph illustrates the error associated with the restrictions embedded in the custom cost function. In the top left quadrant, the performance relative to the restriction of the balance of nodes is displayed, while the top right quadrant details the error present in the Weymouth equation. For its part, the bottom left quadrant reveals the results obtained with regard to the restriction of the compression ratio. Finally, the bottom right compares the differences for each gas network between our proposal and the reference solver. In graphic representations, the letter "R" identifies our model and the "S", the solver employed. }
    \label{fig:ReM1}
\end{figure}

\newpage


\subsection{Percentage of Outliers}
This analysis began by setting a threshold, defining as acceptable those values below 1$\%$. The data exceeding this threshold were then examined. It is essential to note, with regard to the limitation of the balance of nodes, that the values that exceeded the set threshold did not exceed 2$\%$ of the total data set, as illustrated in Figure \ref{fig:ReB2}.
\begin{figure}[H]
    \centering
    \includegraphics[width=1\linewidth]{Cap4/Figures/AtipicosB1.pdf}
    \caption{The figure shows the outliers associated with the node balance equation for the network and solver, each denoted by R and S respectively}
    \label{fig:ReB2}
\end{figure}

In the particular case of the Weymuyh equation, it was observed that these did not exceed 2.5$\%$ of the total data, evidence presented in Figure \ref{fig:ReW2}. This finding underlines the competitiveness of our proposal over conventional methods, highlighting its effectiveness in managing nodal equilibrium restriction.

 
\begin{figure}[H]
    \centering
    \includegraphics[width=1\linewidth]{Cap4/Figures/AtipicosW1.pdf}
    \caption{The figure shows the outliers associated with the Weymouth equation for the grid and solver, each denoted by R and S respectively}
    \label{fig:ReW2}
\end{figure}
In general terms, the behaviour of the outliers is similar to that presented in the previous chapter, with the exception that, when implementing the dynamic restriction, the percentage of data exceeding the threshold experiences an increase of 0.8$\%$ and 0.9$\%$ for the two restrictions, compared with the previous version. This finding underscores a critical aspect that should be considered in future research.


\section{summary}

This section outlines a technique for enhancing the effectiveness of gas distribution networks by employing neural networks. The primary objective of the technique is to enhance flow and pressure management in gas transport through the utilization of a non-linear programming model. The challenges associated with nonlinearity have been successfully addressed, particularly in regard to the intricacies stemming from the Weymouth equation. The efficacy of this approach has been proven in both artificial gas networks and in a particular network within the Colombian context. The results indicate that the proposed model exhibits greater error levels in comparison to the IPOPT program. Nevertheless, in 2.5\% of instances, mistakes surpass the acceptable level of 1\%. This underscores the intrinsic level of competitiveness in our plan.

It's essential to recognize that the training process unfolds independently, without the need for external reference materials. Although the imposition of constraints on unmet demand flows introduces additional costs, which ideally should be modifiable at every phase of the training cycle, a dynamic activation function has been devised. This function effectively mimics the properties of the unmet gas, thus lowering solution expenses, decreasing data fluctuation, and standardizing the median at zero. Consequently, in half of the scenarios analyzed, this method has led to lower expenses than those incurred by traditional techniques, thanks to the consistency of outcomes. Nonetheless, within the unique context of Colombia, there appears to be a minor tendency towards rising budgets. Still, the exceptional time efficiency of neural networks and their capacity to deliver precise outcomes within established margins significantly highlight the advantages of this method over traditional ones.